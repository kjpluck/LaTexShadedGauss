\documentclass[11pt]{article}

\usepackage{shadedgauss}

\begin{document}

%\pgfplotsset{every axis/.append style={font=\tiny}}

\title{Shaded Gauss}
\author{Kevin Pluck\\kevinjpluck@gmail.com}
\date{\vspace{-5ex}} %Sticking in negative space removes gap for date in title!
\maketitle

\section{Usage}
\begin{verbatim}
\midshadedgauss{mean}{std.dev}{xmin}{xmax}{start of shaded area}{end of shaded area}
\leftshadedgauss{mean}{std.dev}{xmin}{xmax}{end of shaded area}
\rightshadedgauss{mean}{std.dev}{xmin}{xmax}{start of shaded area}

\end{verbatim}

\section{Examples}
A good rule of thumb for xmin and xmax is mean +/- 2.5*std.dev.
\subsection{Left shaded gauss}
	\leftshadedgauss{137}{15}{100}{175}{145}
\begin{verbatim}
	\leftshadedgauss{137}{15}{100}{175}{145}
\end{verbatim}
\subsection{Right shaded gauss}
	\rightshadedgauss{137}{15}{100}{175}{129}
\begin{verbatim}
	\rightshadedgauss{137}{15}{100}{175}{129}
\end{verbatim}
\subsection{Mid shaded gauss}
	\midshadedgauss{137}{15}{100}{175}{129}{145}
\begin{verbatim}
	\midshadedgauss{137}{15}{100}{175}{129}{145}
\end{verbatim}



\end{document}  